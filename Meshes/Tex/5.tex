\documentclass[12pt,a4paper]{scrartcl}
\usepackage[utf8]{inputenc}
\usepackage[english,russian]{babel}
\usepackage{indentfirst}
\usepackage{misccorr}
\usepackage{graphicx}
\usepackage{amsmath}
\usepackage{amssymb}
\begin{document}
\section{Потенциалы и заряды системы концентрических проводящих сфер}

Пусть имеется система проводящих концентрических сфер с радиусами 
$r_1<r_2<\dots<r_N$ и зарядами $q_i$. Тогда потенциал внешней сферы по теореме 
Остроградского-Гаусса определяется как
\begin{equation}
  \varphi_N=\frac{\sum\limits_{i=1}^Nq_i}{r_N}\,.
\end{equation}

Потенциал $N-1$ сферы определяется как
\begin{equation}
  \varphi_N=\frac{\sum\limits_{i=1}^{N-1}q_i}{r_{N-1}}+C_{N-1}\,,
\end{equation}
где константа $C_{N-1}$ определяется из условия
\begin{equation}
 \frac{\sum\limits_{i=1}^{N-1}q_i}{r_{N}}+C_{N-1}=
 \frac{\sum\limits_{i=1}^Nq_i}{r_N},\;
  C_{N-1}=\frac{q_N}{r_N}
\end{equation}

Таким образом, потенциалы и заряды на $N$ концентрических сферах определены 
равенством:
\begin{equation}
  \left(
  \begin{array}{c}
    \vec \varphi_1\\
    \vdots\\
    \vec \varphi_N
  \end{array}
  \right)=
  \left(
  \begin{array}{ccccccc}
1/r_1 & 1/r_2 & 1/r_3 & \ldots & 1/r_{N-2} & 1/r_{N-1} & 1/r_N\\
1/r_2 & 1/r_2 & 1/r_3 & \ldots & 1/r_{N-2} & 1/r_{N-1} & 1/r_N\\
1/r_3 & 1/r_3 & 1/r_3 & \ldots & 1/r_{N-2} & 1/r_{N-1} & 1/r_N\\
\vdots & \vdots & \vdots & \ldots & \vdots & \vdots & \vdots\\
1/r_{N-2} & 1/r_{N-2} & 1/r_{N-2} & \ldots & 1/r_{N-2} & 1/r_{N-1} & 1/r_N\\
1/r_{N-1} & 1/r_{N-1} & 1/r_{N-1} & \ldots & 1/r_{N-1} & 1/r_{N-1} & 1/r_N\\    
1/r_N & 1/r_N & 1/r_N & \ldots & 1/r_N & 1/r_N & 1/r_N\\
  \end{array}
  \right)
  \left(
  \begin{array}{c}
    \vec q_1\\
    \vdots\\
    \vec q_N
  \end{array}
  \right)\,.
\end{equation}
Элементы потенциальной матрицы в этой системе определяются как
\begin{equation}
  S_{ij}=
  \left\{
  \begin{array}{c}
    1/r_i\;\mbox{если}\;i>j,\\
    1/r_j\;\mbox{если}\;i\le j.
  \end{array}
  \right.
\end{equation}

\section{Изменение свободной энергии системы}

Изменение электростатической энергии запишем в виде
\begin{equation}
  dU = 
  \frac{\vec{\varphi_{f}^T}\vec{q}_{f}-\vec{\varphi_{i}^T}\vec{q}_{i}}{2}=
  \frac{(\vec{\varphi_{i}^T}+\vec{d\varphi^T})
  (\vec{q}_{i}+d\vec{q})
  -\vec{\varphi_{i}^T}\vec{q}_{i}}{2}
\end{equation}
Заметим, поскольку электроды поддерживаются при постоянном потенциале, то 
вектор изменения потенциалов всех элементов 
\begin{equation}
  d\vec{\varphi}=
  \vec{\varphi}_{\mbox{f}}-\vec{\varphi}_{\mbox{i}}=
  \left(
  \begin{array}{c}
   \vec 0_e\\
   d\vec \varphi_d
  \end{array}
  \right)
\end{equation}
Вектор изменения зарядов всех элементов 
\begin{equation}
  d\vec{q}=
  \vec{q}_{\mbox{f}}-\vec{q}_{\mbox{i}}=
  \left(
  \begin{array}{c}
   d\vec q_{ind}+d\vec q_{tun}\\
   d\vec q_d
  \end{array}
  \right)=
  \left(
  \begin{array}{c}
   d\vec q_{ind}+ed\vec n_{tun}\\
   ed\vec n_d
  \end{array}
  \right)
\end{equation}

Выражение для изменения свободной энергии можно переписать в следующем 
виде с учетом неизменности электрических потенциалов электродов
\begin{equation}
  dU = \frac{\vec{\varphi}^T_{\mbox{i}}d\vec{q}+
  \vec{q}^T_{\mbox{i}}d\vec{\varphi}+
  d\vec{q}^Td\vec{\varphi}}{2}=
  \frac{(\vec{q}^T_d+d\vec{q}^T_d)d\vec{\varphi}_d+
  \vec{\varphi}_d^Td\vec{q}_d+
  \vec{\varphi}_e^Td\vec{q}_{ind}}{2}
\end{equation}
Для нахождения $d\vec q_{ind}$, $\vec{\varphi}_d$ и 
$d\vec{\varphi}_d$ запишем систему уравнений
\begin{equation}
  \left(
  \begin{array}{c}
   \vec q_{ind}\\
   \vec q_d
  \end{array}
  \right)=
  \left(
  \begin{array}{cc}
   \hat C_{ee} & \hat C_{ed}\\
   \hat C_{de} & \hat C_{dd}
  \end{array}
  \right)
  \left(
  \begin{array}{c}
   \vec \varphi_e\\
   \vec \varphi_d
  \end{array}
  \right)
\end{equation}
откуда получаем
\begin{equation}
  \begin{array}{l}
   \vec \varphi_d=\hat C^{-1}_{dd}
   (\vec{q}_d-\hat C_{de}\vec \varphi_e)\\
   \vec q_{ind}=(\hat C_{ee}-\hat C_{ed}\hat C^{-1}_{dd}\hat C_{de})\varphi_e+
   \hat C_{ed}\hat C^{-1}_{dd}\vec q_d.
  \end{array}
\end{equation}
На основе этих уравнений для $d\vec q_{ind}$ и 
$d\vec{\varphi}_d$ получаем:
\begin{equation}
  \begin{array}{l}
   d\vec \varphi_d=\hat C^{-1}_{dd}d\vec{q}_d\\
   d\vec q_{ind}=
   \hat C_{ed}\hat C^{-1}_{dd}d\vec q_d.
  \end{array}
\end{equation}
Получаем для изменения электростатической энергии
\begin{equation}
  dU = 
  \frac{(\vec{q}^T_d+d\vec{q}^T_d)\hat C^{-1}_{dd}d\vec{q}_d+
  (\vec{q}_d^T-\vec \varphi_e^T\hat C_{ed})\hat C^{-1}_{dd}d\vec{q}_d+
  \vec{\varphi}_e^T\hat C_{ed}\hat C^{-1}_{dd}d\vec q_d}{2}
\end{equation}
Окончательно изменения электростатической энергии:
\begin{equation}
  dU = \vec{q}^T_d\hat C^{-1}_{dd}d\vec{q}_d+
  \frac{d\vec{q}^T_d\hat C^{-1}_{dd}d\vec{q}_d}{2}=
  e^2\vec{n}^T_d\hat C^{-1}_{dd}d\vec{n}_d+
  \frac{e^2}{2}d\vec{n}^T_d\hat C^{-1}_{dd}d\vec{n}_d
\end{equation}
Изменение свободной энергии системы включает в себя работу источников
\begin{equation}
 dF=dU + U=e^2\vec{n}^T_d\hat C^{-1}_{dd}d\vec{n}_d+
  \frac{e^2}{2}d\vec{n}^T_d\hat C^{-1}_{dd}d\vec{n}_d+
  e\vec \varphi_e^T(\hat C_{ed}\hat C^{-1}_{dd}d\vec{n}_d+d\vec{n}_{tun})
\end{equation}
\end{document}
